\subsection{Принятые предположения}

\begin{itemize}
\item Каждый документ --- это просто набор слов. Порядок слов и грамматическая роль слов (субъект, объект, глаголы и т.д.) в модели не учитываются.

\item Такие слова, которыми, к примеру, в английском языке, являются \textit{am}, \textit{is}, \textit{are},  \textit{of}, \textit{a}, \textit{the}, \textit{but} и др. не несут никакой информации о темах и поэтому могут быть удалены из документов на этапе предварительной обработки. Фактически, мы можем удалить слова, которые встречаются как минимум в $80-90 \%$ документов, не теряя при этом в итоговом результате.

Например, если наш корпус содержит только медицинские документы, такие слова, как \textit{человек}, \textit{тело}, \textit{здоровье} могут присутствовать в большинстве документов и, следовательно, могут быть удалены, поскольку они не добавляют никакой конкретной информации, которая описывала бы документ.

\item Заранее известно, на какое количество тем будет происходить распределение слов.

\item На каждом шаге алгоритма, при рассмотрении текущего слова, распределения всех предыдущих слов на темы верны, а обновление новым словом происходит с использованием уже имеющейся текущей модели.

\end{itemize}