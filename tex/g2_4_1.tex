\subsection{Ключевые идеи}

\begin{itemize}

\item Квадрат евклидова расстояния

Наиболее часто используемое расстояние в методе $k$-средних --- это квадрат евклидова расстояния. Пример расстояния между двумя точками $x$ и $y$ в $m$-мерном пространстве:

$$d(x, y)^2 = \sum\limits_{j=1}^m (x_j - y_j)^2 = ||x - y||^2_2$$

Здесь индекс $j$ --- это $j$-я координата точек $x$ и $y$.
\item Кластерная инерция

Кластерная инерция --- это имя, данное сумме квадратов ошибок (\textit{Sum of Squared Errors}, \textit{SSE}) в контексте кластеризации, которое представляется следующим образом:

$$SSE = \sum\limits_{i=1}^n\sum\limits_{j=1}^k w^{(i, j)} \cdot ||x^{(i)} - \mu^{(j)}||_2^2$$

Где $\mu^{(j)}$ --- это центроид (среднее значение) для кластера $j$, а $w^{(i, j)}$ равно $1$, если $x^{(i)}$ находится в кластере $j$, и $0$ в противном случае.

Метод $k$-средних можно понимать как алгоритм, который пытается минимизировать инерции кластеров.

\end{itemize}
