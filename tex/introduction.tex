\chapter*{Введение}

С ростом доступности электронных документов и быстрым ростом всемирной паутины задача автоматической категоризации документов стала ключевым cпособом классификации и группирования информации и знаний любого рода. Для правильной классификации электронных документов, онлайн-новостей, блогов, электронной почты и электронных библиотек необходимы интеллектуальный анализ текста (англ. \textit{Text Mining}), машинное обучение (англ. \textit{Machine Learning}) и методы обработки текстов на естественном языке (англ. \textit{Natural Language Processing, NLP}).

Современные системы обработки текстов на естественном языке
могут анализировать неограниченные объемы текстовых данных. Они могут понимать суть сложных контекстов, расшифровывать двусмысленности языка, извлекать ключевые факты и взаимосвязи. Учитывая огромное количество неструктурированных данных, которые создаются каждый день, от электронных медицинских карт до сообщений в социальных сетях, обработка текстов на естественных языках стала критически важной для эффективного анализа текстовых данных.


\addcontentsline{toc}{chapter}{Введение}
\clearpage
