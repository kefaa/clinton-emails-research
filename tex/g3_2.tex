\section{Предобработка электронных писем корпорации Enron}

\subsection{Выделение метаданных из сырого текста писем}

Набор данных \textit{Enron} также требует предварительной обработки. К примеру, так выглядит необработная информация одного электронного письма:

\begin{verbatim}
Message-ID: <18782981.1075855378110.JavaMail.evans@thyme>
Date: Mon, 14 May 2001 16:39:00 -0700 (PDT)
From: phillip.allen@enron.com
To: tim.belden@enron.com
Subject: 
Mime-Version: 1.0
Content-Type: text/plain; charset=us-ascii
Content-Transfer-Encoding: 7bit
X-From: Phillip K Allen
X-To: Tim Belden <Tim Belden/Enron@EnronXGate>
X-cc: 
X-bcc: 
X-Folder: \\Phillip_Allen_Jan2002_1\\Allen, Phillip K.\\\'Sent Mail
X-Origin: Allen-P
X-FileName: pallen (Non-Privileged).pst

Here is our forecast
\end{verbatim}

Конечно, анализировать данные (в том числе метаинформацию о письме) в таком формате бессмысленно. Для обработки мы будем использовать библиотеку \textit{email} \cite{bib8}. Данная библиотека позволяет из сырых данных выделить вспомогательную информацию о письме, в частности, библиотека позволяет выделить следующие интересные нам атрибуты:
\begin{itemize}
\item полное содержание письма,
\item дата отправки,
\item адрес получателя,
\item адрес отправителя,
\item тема письма,
\item логин отправителя.
\end{itemize}

\subsection{Выделение содержания писем}

После этого требуется также привести содержание письма в приемлимый для дальнейшего обучения вид. Например, для письма с содержанием ниже мы хотим выделить только единицы, имеющие отношение к сути письма.

\begin{verbatim}

Forwarded by Phillip K Allen/HOU/ECT on 09/12/2000 11:22 AM 

Michael Etringer

09/11/2000 02:32 PM

To: Phillip K Allen/HOU/ECT@ECT
cc:  
Subject: Contact list for mid market

Phillip,
Attached is the list. Have your people fill in the columns 
highlighted in yellow. As best can we will try not to overlap
on accounts. 
Thanks, Mike
\end{verbatim}

Выделение этих единиц происходит в соответствии со следующей последовательности шагов:
\begin{enumerate}
\item Перевод всех символов в нижний регистр.
\item Удаление всех слов, содержащих цифры. Такие слова не несут смысловой нагрузки и, соответственно, влияют на качество обучения в худшую сторону (подавляющее большинство таких слов было получено отправителями по ошибке).
\item Удаление единиц, соответствующих информации о пересланных письмах.
\item Удаление единиц, соответствующих информации о вложениях в письме.
\item Удаление единиц, соответствующих почтовым адресам, присутствующим в тексте писем.
\item Удаление единиц, соответствующих корпоративным именам пользователей.
\item Удаление единиц, соответствующих ссылкам в сети Интернет.
\item Удаление единиц, не несущих смысловой составляющей, в заголовке письма.
\end{enumerate}

Шаги 3-7 выполняются с использованием регулярных выражений. Регулярные выражения являются стандартом сопоставления шаблонов для анализа и замены строк. В языке программирования 
$Python$ регулярные выражения реализованы в библиотеке $re$, с помощью неё получилось
довольно компактно и быстро <<почистить>> электронные письма:

\begin{verbatim}
# text inside <>, some forwarded info
text = re.sub('<.*>', '', text)

# text inside [], some attachements
text = re.sub('\[.*\]', '', text) 

# remove email addresses
text = re.sub('\S*@\S*\s?', '', text) 

# some corporation usernames
text = re.sub('\S*/hou/\S*s?', '', text)

# remove links
text = re.sub('http[s]?://\S+', '', text) 

# remove links
text = re.sub('www\.\S+', '', text) 
\end{verbatim}

Далее нужно выделить основной текст письма (часть после \texttt{Subject: }), это можно
сделать при помощи соответствующего регулярного выражения:

\begin{verbatim}
text = re.sub('.*subject:', '', text)
\end{verbatim}

Однако, используя библиотеку $re$, такой вариант регулярного выражения работает слишком долго, поэтому пришлось реализовать его самостоятельно:

\begin{verbatim}
msg_text = 'subject:'
idx = text.find(msg_text)
if idx != -1:
    text = text[idx + len(msg_text):].strip()
\end{verbatim}

Такой вариант на имеющихся данных работает в десятки раз быстрее. Таким образом, теперь
всё готово для быстрой обработки текста.

В результате, для примера выше, дальнейшая работа будет производиться со следующим текстом:
\begin{verbatim}
contact list for mid market. phillip, attached is the list. 
have your people fill in the columns highlighted in yellow. 
as best can we will try not to overlap on accounts. thanks, mike'
\end{verbatim}

