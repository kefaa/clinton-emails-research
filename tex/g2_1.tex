\section{Тематическое моделирование}

Тематическое моделирование --- это метод классификации документов, аналогичный кластеризации по числовым данным, который находит некоторые естественные группы элементов (темы), даже в случаях, когда пользователь не имеет конкретной цели касательно нахождения определенных тем.

Документ может быть частью нескольких тем, как в нечеткой (мягкой) кластеризации, в которой каждый элемент данных принадлежит более чем одному кластеру.

Тематическое моделирование предоставляет методы для автоматической организации, понимания, поиска и обобщения больших электронных архивов. Это может помочь в следующих случаях:

\begin{itemize}
\item обнаружение скрытых (неочевидных) тем в корпусе (т.е. общем наборе слов) данных,
\item классификация документов по обнаруженным темам,
\item использование классификации для, собственно, организации, обобщения либо поиска интересующих документов.
\end{itemize}

Например, предположим, что документ относится к темам \textit{еда}, \textit{собаки} и \textit{здоровье}. Таким образом, если пользователь запрашивает <<\textit{корм для собак}>>, вышеупомянутый документ может определиться как релевантный, поскольку, помимо других тем, он охватывает и эти темы. Другими словами, его релевантность по отношению к запросу может быть выяснена даже без просмотра всего документа, а только на основании уже известных тем.

Получается, аннотируя документ на основе тем, предсказанных методом моделирования, становится возможно оптимизировать выполняемый процесс поиска.