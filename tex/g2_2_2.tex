\subsection{Поиск слов, репрезентативных по отношению теме}

Чтобы описать, какие слова описывают тему, можно поступить одним из следующих способов.

\begin{itemize}
\item Сортировка слов в зависимости от их вероятностей.

Из каждой темы выбирается какое-то количество лучших слов для представления темы. Этот шаг не всегда может быть необходим, потому что, если корпус небольшой, можно хранить все слова, отсортированные по их вероятностям.	

\item Установление \textit{порога} вероятностей. 

Все слова в теме, получившие вероятность выше порогового значения, могут быть использованы как ее представители (в порядке сортировки).
\end{itemize}