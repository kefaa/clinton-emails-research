\subsection{Исследования над письмами корпорации Enron}

По крайней мере два исследования, касающихся классификации текстов, были выполнены на наборе данных \textit{Enron}. Один из них --- автоматическая категоризация электронной почты по папкам, выполняемая факультетом компьютерных наук Массачусетского университета \cite{bib_5}. Другой связан с анализом социальных сетей. Используя корпус \textit{Enron} и судебные документы, выпущенные судом США по делам о банкротстве, Джитеш Шетти и Джафар Адиби извлекли из электронной почты социальную сеть, состоящую из 151 сотрудника, соединив людей, которые обменивались электронными письмами \cite{bib_6}.

Кроме того, команда \textit{Legal Track} конференции по извлечению текстов (<<\textit{Text Retrieval Conference Legal Track}>>) фокусировалась на методах крупномасштабного поиска текста. Команда ученых изучала следующие методы поиска: булевы, нечеткие модели поиска, вероятностные (байесовские) модели, статистические методы, подходы машинного обучения, инструменты категоризации и анализ социальных сетей. Исследователи \textit{TREC Legal Track}
пришли к выводу, что «всего от 22 до 57 процентов релевантных документов могут быть извлечены с помощью различных альтернативных методов поиска».

В статье Шетти и Адиби набор данных \textit{Enron} использовался для классификации электронной почты на основе моделирования энтропии графа. Энтропия пыталась выбрать наиболее интересные вершины в графе, вершины которого представляют электронные письма, а ребра --- сообщения между пользователями \cite{bib_7}.

Что касается непосредственно тематического моделирования и кластеризации, проводимых в данной работе, основная масса исследований представлена небольшими проектами, выполненными пользователями \textit{Kaggle}. Как и в случае с электронными письмами Клинтон, в силу аналогичных причин, качество данных работ не является высоким.
