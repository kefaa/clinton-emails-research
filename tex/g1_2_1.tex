\subsection{Исследования над письмами Хиллари Клинтон}


Анализ, проведенный центром Беркмана Кляйна по интернету и обществу в Гарвардском университете и центром Шоренштейна в Гарвардской школе Кеннеди, показывает, что споры по поводу электронной почты Клинтона получили больше освещения в основных средствах массовой информации, чем любая другая тема во время президентских выборов США 2016 года \cite{clinton_research_bib}.

Например, исследование профессора Уэйн Олдфорда и студента бакалавра университета Уотерлу построили инструмент для анализа содержимого писем \cite{clinton_oldford_bib}.

Инструмент предоставляет визуальные аналитические инструменты и демонстрирует, как много можно узнать о человеке из того, что ошибочно считается неинформативными метаданными. По словам Олдфорда, <<общественность может воспроизвести наши анализы и сама увидеть, как они могут быть показательными, особенно в сочетании с другими общедоступными источниками>>.

Например, приложение показывает ежедневный объём электронной почты, отправленной и полученной Клинтон, а также слова, наиболее часто встречающиеся в электронных письмах за выбранный период времени.

Например, при анализе данных исследователи обнаружили 10 периодов отсутствия электронных писем от Клинтон, пока она была госсекретарем, включая значительный разрыв между 30 октября и 9 ноября 2012 года, который совпадает с первоначальным расследованием нападения в Бенгази и его последствий, а также президентских выборов в США в 2012 году.

Существует большое количество исследований электронных писем Клинтон, выполненные пользовалями $Kaggle$. К сожалению, качество данных работ не является высоким, поскольку они, во-первых, не являются большими проектами, а представляют собой скорее любительские исследования, а во-вторых, при этом часто выполнены с недостатками, например, с плохой предобработкой исходных данных. 
