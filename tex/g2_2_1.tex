\subsection{Определение модели}

Допустим, имеется несколько документов, каждый из которых содержит присутствующие в нем слова, отсортированные по частоте встречаемости. Целью метода является определение соответствий между словами и темами. Это можно отобразить, например, с помощью таблицы ниже. Каждая строка в таблице представляет отдельную тему, а каждый столбец --- отдельное слово в корпусе. Каждая ячейка содержит вероятность того, что слово (столбец) принадлежит теме (строке).

$ $

\begin{table}[h]
\centering
\begin{tabular}{ | l | l | l | l | l | l |}
\hline
 & \textit{слово}$_1$ & \textit{слово}$_2$  & \textit{слово}$_3$  & \textit{слово}$_4$  & \ldots \\ \hline
\textit{тема}$_1$ & $0.01$ & $0.23$ & $0.19$ & $0.03$ & \\ \hline 
\textit{тема}$_2$ & $0.21$ & $0.07$ & $0.48$ & $0.02$ & \\ \hline 
\textit{тема}$_3$ & $0.53$ & $0.01$ & $0.17$ & $0.04$ & \\ \hline 
\ldots & & & & & \\ \hline 
\end{tabular}
\caption{Пример представления модели латентного размещения Дирихле}
\end{table}