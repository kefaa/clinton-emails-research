\section{Построение \textit{валидной области}}

Для нахождения \textit{валидной области}, мы используем подход, использующий начальный треугольник, найденный ранее.
Также, вместо того, чтобы заранее разбивать все пересекающиеся треугольники, мы будем разбивать только те частично валидные треугольники, которые появлялись в ходе работы дальнейшего алгоритма, чтобы не совершать лишние операции над невалидными треугольниками. Подробнее, алгоритм построения \textit{валидной области} состоит из следующих шагов:

\begin{enumerate}
\item
Каждый треугольник может иметь одно из следующих состояний:
\begin{itemize}
\item непосещенный треугольник,
\item валидный треугольник,
\item частично валидный треугольник.
\end{itemize}
Изначально все треугольники помечаются как непосещенные.

\item
Начальный треугольник, найденный ранее, помечается валидным и добавляется в некоторое множество $\mathbf{S}$.

\item
Если множество $\mathbf{S}$ пусто, перейти к пункту 5. Иначе, достать треугольник $T$ из $\mathbf{S}$.

\item
Для каждого непосещенного треугольника $T_a$, смежного с $T$, если $T_a$ не пересекается с другими треугольниками, он добавляется в $\mathbf{S}$. В противном случае, $T_a$ помечается частично валидным и добавлется в другое множество $\mathbf{P}$. $T_a$ добавляется в $\mathbf{P}$ вместе с информацией о ребре $e_p$, которое являлось смежным для $T$ и $T_a$.

\item
Если множество $\mathbf{P}$ пусто, перейти к пункту 7. Иначе, достать частично валидный треугольник $T_p$ и его ребро $e_p$ из $\mathbf{P}$.

\item
Триангулировать $T_p$ (детали в секции 2.5). Продолжить построение \textit{валидной области} с помощью $T_p$ и его подтреугольников (как описано в секции 2.6). Если другой начальный треугольник найден, он добавляется в $\mathbf{S}$ и нужно перейти к пункту 2. Иначе, перейти к пункту 4.

\item Построение \textit{валидной области} завершено.

\end{enumerate}

