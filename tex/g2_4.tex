\section{Метод $k$-средних}

Алгоритм $k$-средних нацелен на поиск и группировку в классы точек данных, которые имеют большое сходство между собой. В терминах алгоритма это сходство понимается как противоположность расстояния между точками данных. Чем ближе точки данных, тем больше они будут похожи и с большей вероятностью принадлежат одному кластеру \cite{bib_9}.

Алгоритм $K$-средних чрезвычайно прост в реализации и очень эффективен с точки зрения вычислений. Это основные причины, по которым он так популярен. Однако они не очень хороши для идентификации классов при работе с группами, не имеющими сферической формы распределения.

\subsection{Ключевые идеи}

\begin{itemize}

\item Квадрат евклидова расстояния

Наиболее часто используемое расстояние в методе $k$-средних --- это квадрат евклидова расстояния. Пример расстояния между двумя точками $x$ и $y$ в $m$-мерном пространстве:

$$d(x, y)^2 = \sum\limits_{j=1}^m (x_j - y_j)^2 = ||x - y||^2_2$$

Здесь индекс $j$ --- это $j$-я координата точек $x$ и $y$.
\item Кластерная инерция

Кластерная инерция --- это имя, данное сумме квадратов ошибок (\textit{Sum of Squared Errors}, \textit{SSE}) в контексте кластеризации, которое представляется следующим образом:

$$SSE = \sum\limits_{i=1}^n\sum\limits_{j=1}^k w^{(i, j)} \cdot ||x^{(i)} - \mu^{(j)}||_2^2$$

Где $\mu:{(j)}$ --- это центроид (среднее значение) для кластера $j$, а $w^{(i, j)}$ равно $1$, если $x^{(i)}$ находится в кластере $j$, и $0$ в противном случае.

Метод $k$-средних можно понимать как алгоритм, который пытается минимизировать инерции кластеров.

\end{itemize}

\subsection{Алгоритм}

\begin{enumerate}

\item Сперва нужно выбрать $k$ --- количество кластеров, которые мы хотим найти.

\item Затем алгоритм случайным образом выберет центроиды каждого кластера.

\item Для каждой точки будет вычислен выбран ближайший центроид (с использованием евклидова расстояния).

\item Вычисляется инерция кластера.

\item Новые центроиды будут вычисляться как среднее значение точек, принадлежащих центроиду предыдущего шага.

\item Вернуться к шагу 3.

\end{enumerate}

\subsection{Гиперпараметры алгоритма}

\begin{itemize}
\item Количество кластеров: количество создаваемых кластеров и центроидов.
\item Максимальное количество итераций: количество итераций, которое будет пытаться совершить алгоритм за один запуск.
\item Количество случайных запусков: количество запусков алгоритма с разными начальными значениями центроидов. Конечным результатом работы алгоритма будет лучший (с точки зрения инерции) из всех запусков.
\end{itemize}

\subsection{Проблемы алгоритма}

\begin{itemize}
\item Выходные данные работы алгоритма не всегда будут одинаковыми, потому что начальные центроиды устанавливаются случайным образом, соответственно, это будет влиять на весь процесс алгоритма.

\item Как было сказано ранее, из-за природы евклидова расстояния этот алгоритм не подходит для работы с кластерами, которые принимают несферические формы.
\end{itemize}