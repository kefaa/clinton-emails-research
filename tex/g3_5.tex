\section{Триангуляция}

Частично валидный треугольник $T_p$ имеет некоторое количество отрезков пересечений с другими треугольниками. Триангуляция на этом шаге выполняется для того, чтобы: (1) разделить $T_p$ на подтреугольники, которые содержат отрезки пересечений в качестве своих сторон, (2) продолжить построение внутри сетки из полученных подтреугольников.

Подробнее, шаги для (1) следующие:

\begin{enumerate}
\item Разделить каждую сторону треугольника $T_p$ всеми отрезками пересечений $\{s_i\}$.
\item Разделить каждый $s_i$ всеми другими отрезками пересечений.
\item Триангулировать $T_p$ с помощью двумерной триангуляции Делоне с ограничениями (\textit{2D constrained Delaunay triangulation}) \cite{triangulation} вместе со сторонами треугольника и отрезками пересечений, полученных в $1$ и $2$.

\end{enumerate}
Для полученных подтреугольников, дальнеший алгоритм построения начинается с ребра $e_p$. Подтреугольник, смежный ребру $e_p$ помечается валидным и становится начальным для процесса построения в подтреугольниках. Затем валидная часть $T_p$ распространяется к соседним подтреугольникаам до тех пор, пока не достигнет ребер, являющихся частью отрезков пересечений, которые играют роль ребер-входов в соседние треугольники $T_c$ для следующего шага, описанного в секции 2.6.


