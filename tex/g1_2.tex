\section{Электронные письма корпорации Enron}

Enron являлась государственной корпорацией штата Орегон со штаб-квартирой в городе Хьюстон. До объявления о банкротстве в декабре 2001 года Enron была седьмой по величине корпорацией в США. В феврале 2002 года Федеральная комиссия по регулированию в области энергетики (Federal Energy Regulatory Commission, FERC) начала всестороннее расследование торговой деятельности Enron на рынках электроэнергии Калифорнии. Согласно FERC, Enron получила некоторую информацию о рынке, недоступную для ее конкурентов. Прибыль Enron превысила 500 миллионов долларов в 2000 и 2001 годах. Расследование пришло к выводу, что многие торговые стратегии, используемые Enron, нарушают рыночные отношения, утвержденные Федеральной комисиией для Калифорнии. С июня 2002 года Министерство юстиции США возбудило уголовные дела против 30 человек, включая Джеффри Скиллинга, бывшего президента и генерального директора Enron, а также других руководителей высшего звена. Обвинения включали сговор, мошенничество с ценными бумагами и инсайдерскую торговлю.

Исходный набор данных Enron был обнародован и размещен в Интернете Федеральной комиссией во время расследования энергетического кризиса в Западной Европе 2000-2001 годов. Позднее набор данных был приобретен Лесли Кельблинг из Массачусетского технологического института, где было выявлено несколько проблем с целостностью данных. Вскоре после этого группа исследователей из SRI International, некоммерческой корпорации, основанной Стэнфордским университетом, во главе с Мелиндой Гарвасио провела серьезную очистку и удаление вложений и отправила ее профессору Уильяму Коэну из Университета Карнеги-Меллона, который сделал публикацию на своей веб-странице. В документе с анализом базы данных Enron, представленном на конференции 2004 года, делается вывод о том, что набор данных Enron «подходит для оценки методов классификации электронной почты».

Используемый набор данных состоит из архивных электронных писем сотрудников Enron, в основном  руководителей высшего звена и трейдеров, из которых были удалены вложения. Также некоторые из писем были удалены по требованию сотрудников корпорации. Общее количество писем -- более полумиллиона.