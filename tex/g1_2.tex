\section{Исследования, проведенные над наборами данных}

По крайней мере два исследования, касающихся классификации текстов, были выполнены на наборе данных Enron. Один из них --- автоматическая категоризация электронной почты по папкам, выполняемая факультетом компьютерных наук Массачусетского университета \cite{bib_5}. Другой связан с анализом социальных сетей. Используя корпус Enron и судебные документы, выпущенные судом США по делам о банкротстве, Джитеш Шетти и Джафар Адиби извлекли из электронной почты социальную сеть, состоящую из 151 сотрудника, соединив людей, которые обменивались электронными письмами \cite{bib_6}.

Кроме того, команда Legal Track конференции по извлечению текстов (<<Text Retrieval Conference Legal Track>>) фокусировалась на методах крупномасштабного поиска текста. Команда ученых изучала следующие методы поиска: булевы, нечеткие модели поиска, вероятностные (байесовские) модели, статистические методы, подходы машинного обучения, инструменты категоризации и анализ социальных сетей. Исследователи TREC Legal Track пришли к выводу, что «всего от 22 до 57 процентов релевантных документов могут быть извлечены с помощью различных альтернативных методов поиска».

В статье Шетти и Адиби набор данных Enron использовался для классификации электронной почты на основе моделирования энтропии графа. Энтропия пыталась выбрать наиболее интересные вершины в графе, вершины которого представляют электронные письма, а ребра --- сообщения между пользователями \cite{bib_7}.

Что касается непосредственно тематического моделирования и кластеризации, проводимых в данной работе, основная масса исследований представлена небольшими проектами, выполненными пользователями \textit{Kaggle}. К сожалению, качество данных работ не является высоким, поскольку они, во-первых, не являются большими проектами, а представляют собой скорее любительские исследования, а во-вторых, при этом часто выполнены с недостатками, например, с плохой предобработкой исходных данных. 