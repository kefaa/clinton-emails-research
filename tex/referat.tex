\chapter*{РЕФЕРАТ}

Дипломная работа, 49 с., 26 рис., 8 таблиц, 19 источников.

Ключевые слова: ОБРАБОТКА ТЕКСТОВ, МАШИННОЕ ОБУЧЕНИЕ, ЛАТЕНТНОЕ РАЗМЕЩЕНИЕ ДИРИХЛЕ, МЕТОДЫ ПОНИЖЕНИЯ РАЗМЕРНОСТИ, ВЕКТОРНОЕ ПРЕДСТАВЛЕНИЕ ТЕКСТОВ, СТАТИСТИЧЕСКИЙ АНАЛИЗ, МЕТОДЫ КЛАСТЕРИЗАЦИИ, НЕЙРОННЫЕ СЕТИ.

\vspace{1.5 ex}
Объект исследования --- наборы данных деловых электронных переписок. 

Цель работы --- анализ деловых электронных переписок методами машинного обучения.

Методы исследования --- латентное размещение Дирихле, методы кластеризации, методы обработки текстов, методы понижения размерности, методы получения векторных представлений.

Работа посвящена исследованию и анализу деловых электронных переписок, в частности, переписок Хиллари Клинтон и переписок сотрудников корпорации Enron. В результате работы был произведен статистический анализ электронных переписок. Были обнаружены закономерности в исходных данных. Также была разработана кластеризация содержаний электронных писем, в результате которой получились интерпретируемые результаты, что показало эффективность разработанных методов.
 


%%%%%%%%%%%%%%%%%%%%%%%%%%%%%%%%%%%%%%%%%%%%%%%%%%%%%%%%%%%
\chapter*{РЭФЕРАТ}

Дыпломная праца, 49 с., 26 рыс., 8 табліц, 19 крыніц.

Ключавыя словы: АПРАЦОЎКА ТЭКСТАЎ, МАШЫННАЕ НАВУЧАННЕ, ЛАТЭНТНАЕ РАЗМЯШЧЭННЕ ДЫРЫХЛЕ, МЕТАДЫ ЗНІЖЭННЯ ПАМЕРНАСЦІ, ВЕКТАРНАЕ ПРАДСТАЎЛЕННЕ ТЭКСТАЎ, СТАТЫСТЫЧНЫ АНАЛІЗ, МЕТАДЫ КЛАСТАРЫЗАЫІ, НЕЙРОНАВЫЯ СЕТКІ.

Аб'ект даследавання --- наборы дадзеных дзелавых электронных перапісак.

Мэта работы --- аналіз дзелавых электронных перапісак метадамі машыннага навучання.

Метады даследавання --- латэнтнае размяшчэнне Дырыхле, метады кластарызацыі, метады апрацоўкі тэкстаў, метады зніжэння памернасці, метады атрымання вектарных уяўленняў.

Праца прысвечана даследаванню і аналізу дзелавых электронных перапісак, у прыватнасці, перапісак Хілары Клінтан і перапісак супрацоўнікаў карпарацыі Enron. У выніку працы быў выраблены статыстычны аналіз электронных перапісак. Былі выяўлены заканамернасці ў зыходных дадзеных. Таксама была распрацавана кластарызацыя зместаў электронных перапісак, у выніку якой атрымаліся інтэрпрэтаваныя вынікі, што паказала эфектыўнасць распрацаваных метадаў.


%%%%%%%%%%%%%%%%%%%%%%%%%%%%%%%%%%%%%%%%%%%%%%%%%%%%%%%%%%%

\chapter*{ABSTRACT}

Diploma thesis, 49 p., 26 fig., 8 tables, 19 sources.

Keywords: TEXT PROCESSING, MACHINE LEARNING, LATENT DIRICHLET ALLOCATION, DIMENSION REDUCTION METHODS, TEXT EMBEDDINGS, STATISTICAL ANALYSIS, CLUSTERIZATION METHODS, NEURAL NETWORKS.

\vspace{1.5 ex}
The object of research is business e-mail datasets.

Objective: analysis of business e-mails using machine learning methods.

Research methods --- latent Dirichlet allocation, clustering methods, text processing methods, dimension reduction methods, methods for obtaining text embeddings.

The work is devoted to the research and analysis of business e-mails, in particular, the e-mails of Hillary Clinton and the e-mails of employees of the Enron corporation. As a result of the work, a statistical analysis of emails was carried out. Patterns were found in the original data. Also, the clustering of the contents of e-mails was developed, as a result of which interpretable results were obtained, which showed the effectiveness of the developed methods.
