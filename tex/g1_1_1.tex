\subsection{Электронные письма Хиллари Клинтон}

В 2015 году Хиллари Клинтон (американский политик, государственный секретарь США в 2009-2013 гг., кандидат в президенты США в 2016 г.) была вовлечена в большое количество споров по поводу использования личных учетных записей электронной почты на негосударственных серверах во время ее пребывания на посту государственного секретаря США. Некоторые политические эксперты утверждают, что использование Клинтон личных учетных записей электронной почты для ведения дел госсекретаря является нарушением протоколов и федеральных законов, обеспечивающих надлежащий учет деятельности правительства. 



Был подан ряд исков о свободе информации из-за того, что Государственный департамент США не опубликовал полностью электронные письма, отправленные и полученные на личные аккаунты Клинтон.

В июле расследование ФБР пришло к выводу, что никакой <<разумный прокурор>> не будет возбуждать уголовное дело против госпожи Клинтон, но что она и ее помощники <<крайне небрежно>> обращались с секретной информацией. Затем ФБР удивило всех, за 11 дней до выборов, объявив, что изучает недавно обнаруженные электронные письма, отправленные или полученные Хиллари Клинтон. Как позже выяснилось, она установила адреса электронной почты на 
государственном сервере для своего давнего помощника Хумы Абедина и начальника штаба Госдепартамента Шерил Миллс. На сегодняшний день Государственным департаментом США опубликовано почти 7000 страниц отредактированных электронных писем Клинтон. 

Документы были опубликованы в формате \textit{pdf}. Платформа \textit{Kaggle} очистила и нормализовала выпущенные документы и разместила их для публичного анализа \cite{bib1}. Мы будем основываться именно на датасете, опубликованном \textit{Kaggle}.
